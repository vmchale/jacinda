\documentclass{report}

\usepackage{mathpartir}
\usepackage{syntax}
\usepackage{hyperref}

\begin{document}

\title{Foundations and Limitations of Jacinda}
\author {Vanessa McHale}
\maketitle

\tableofcontents

\section{Introduction}

Jacinda is an expression-oriented, functional language that performs a subset of Awk's function.

In many ways it is unsatisfactory or even defective, but it is useful and so I present its

\section{Core Approach}

In order to stand toe-to-toe with Awk, Jacinda has some special facilities to work with streams filtered via regular expressions.

To define a stream from a pattern: {\tt \{\%/Bloom/\}\{`0\}}. {\tt\{`0\}} is a line expression; we could write

\section{Type System}

A signal defect of the type system is that it fails to distinguish expressions in the context of a line from expressions in general. In fact expressions in the context of a line form a monad but are not treated as such because explicit typing would be too burdensome and I did not know how to implement implicit coercions.

Consider:

\begin{verbatim}
[y]|> {|`0~/^$/}
\end{verbatim}

This evaluates whether the last line is blank.

\newcommand\Tilde{\char`\~}
\newcommand\Caret{\char`\^}

In a more principled world, we would have {\tt `0 : Ctx Str} and we would force {\tt \{| ... \}} to take an expression of type {\tt Ctx a}. Then we could do something like {\tt \{|(\Tilde/\Caret\$/)"`0\}}... of course this is more prolix.

\subsection{Typeclasses}

Type classes are used to witness implementations \cite{adhoc}.

\bibliographystyle{plain}
\bibliography{foundations.bib}

\end{document}
